\documentclass{beamer}
\usetheme{Madrid} % My favorite!
%\usetheme{Boadilla} % Pretty neat, soft color.
%\usetheme{default}
%\usetheme{Warsaw}
%\usetheme{Bergen} % This template has nagivation on the left
%\usetheme{Frankfurt} % Similar to the default 
%with an extra region at the top.
%\usecolortheme{seahorse} % Simple and clean template
%\usetheme{Darmstadt} % not so good
% Uncomment the following line if you want %
% page numbers and using Warsaw theme%
% \setbeamertemplate{footline}[page number]
%\setbeamercovered{transparent}
\setbeamercovered{invisible}
% To remove the navigation symbols from 
% the bottom of slides%
\setbeamertemplate{navigation symbols}{} 
%
\usepackage{graphics}
\usepackage{graphicx}
\usepackage{hyperref}
\usepackage{tabularx,multirow}


%\usepackage000000..{bm}         % For typesetting bold math (not \mathbold)
\logo{\includegraphics[height=0.6cm]{imperial.pdf}}
%
\title[Cycling simulation]{Cycling simulation \\or\\ \textit{When} to break away}

\author[]{%
  Dan Demeter\inst{1} \and
  Alexandru Paunoiu\inst{1} \and
  C\'esar Prout\'e \inst{1} \and \\
  Julian Sutherland \inst{1} \and
  Robert K\inst{1}
  }
\institute[Imperial College London]{  
  \inst{1}%
  Imperial College London \\
  Department of Computing \\
  \vspace{1cm}
  Supervised by: Panos PARPAS\\
  
 }  

\begin{document}
%
\begin{frame}
\titlepage
\end{frame}
%
\begin{frame}
\frametitle{ mapM \$whoami [Julian, C\'esar, Robert, Alex, Dan] }

\begin{itemize}
	\item Group 29
	\vspace{1cm}
	\item (Keen) Cyclists
	\item Our organisation
\end{itemize}

\end{frame}

%Outline
\begin{frame}

\frametitle{Outline}

\begin{itemize}
	\item Our Motivation
	\vspace{0.5cm}
	\item Initial problem ( many more to come...)
	\vspace{0.5cm}
	\item Our solution
	\vspace{0.5cm}
	\item Achievements \& what could we have done better 
\end{itemize}

\end{frame}


\begin{frame}
\frametitle{Motivation}

\begin{itemize}
	\item Strategies in races can be crucial
	\item Modelling races can lead to improved performance
	\pause
	\item But this is novel...
\end{itemize}

\end{frame}

\begin{frame}
\frametitle{The solution}

\begin{itemize}
	\item Step 1: Simulation
	\item Step 2: Game theoretical model
\end{itemize}

\end{frame}


\begin{frame}
\frametitle{Step 2: Game theoretical model - Preliminaries}

\textbf{Pay-off matrix}\\
The pay-off each player gets by playing certain strategies is contained in a matrix, we call this the pay-off matrix. By using this matrix, players can choose the best strategy they should play in certain situations.

\begin{example}[Pay-off matrix]
\begin{table}[ht!]
	\hspace{-4em}
	\centering
	\begin{tabular}{ccccc|}
		& & \multicolumn{3}{c}{Player 1}                                \\ \cline{3-5}
		& & A & B & \multicolumn{1}{c}{C}                               \\ \cline{3-5}
		\multirow{3}{*}{Player 2} & \multicolumn{1}{|c|}{a} & 2 & 1 & 7 \\
		& \multicolumn{1}{|c|}{b} & 4 & 5 & 6                           \\
		& \multicolumn{1}{|c|}{c} & 3 & 1 & 7                           \\ \cline{3-5}
	\end{tabular}
\end{table}
\end{example}

\end{frame}

\begin{frame}
\frametitle{Achievements and Contribution}

\begin{itemize}
	\item Implemented the proposed simulation in papers %add details
	\item Combined psychological model with performance model
	\item Game theoretical model which could be valuable for simulating races
\end{itemize}

\end{frame}

\begin{frame}
\frametitle{Recap}

\begin{itemize}
	\item TODO
\end{itemize}

\end{frame}


\begin{frame}
\frametitle{Thank you!}

\huge
\centering Q\&A

\end{frame}

% End of slides
\end{document} 
